% Footnotes used inline in the text.
\newcommand{\manualhandlingfootnote}{\footnote{\url{http://www.hse.gov.uk/pubns/indg143.pdf}}}
\newcommand{\chargingfootnote}{\footnote{\url{https://www.studentrobotics.org/docs/kit/batteries/imax_b6_charger\#ChargingChecklist}}}
\newcommand{\estatesfacilitiesfootnote}{\footnote{Estates and Facilities: \url{http://www.southampton.ac.uk/estates/}}}
\newcommand{\esofootnote}{\footnote{Equipment and Safety Officer; for the duration of this event, this role will be filled by Kier Davis (\email{me@kierdavis.com}).}}


\newcommand{\groupname}{Student Robotics (Southampton)}
\newcommand{\assessorname}{Kier Davis}
\newcommand{\assessoremail}{me@kierdavis.com}
\newcommand{\assessmentdate}{July 6, 2016}


\newcommand{\activityname}{Summer School (competition)}
\newcommand{\activitydate}{August 1 to August 5, 2016}
\newcommand{\activitytime}{All day}
\newcommand{\activitylocation}{Zepler L2 Labs (Building 59)}
\newcommand{\activitysummary}{
    The summer school is run by the ECS department with assistance from
    Student Robotics Southampton. The participants in the summer school are
    invited to design, build and test an autonomous robot in small teams. The
    robots must perform a simple task, usually involving locating and moving
    coloured boxes, which is performed competitively against other teams'
    robots.

    In addition to this main event, there are a number of taught laboratory
    sessions. Members of Student Robotics Southampton will mentor and provide assistance to
    the participants, as well as set up the arena for the competition and lead
    the laboratory sessions.

    This risk assessment specifically covers the use of The Cube for assembly
    and use of an arena for the competition matches, as well as continuation of
    the construction of the robots. Manual and/or power tools will be used to
    assemble the arena and robot chassis, and lithium polymer batteries may be
    used inside the robots.

    This document should be considered an extension of the general risk
    assessment for the summer school\footnote{Available at
    \url{http://dl.kierdavis.com/srobo-soton-risk-assessments/latest/20160801-summer-school.pdf}}.
}


\newcommand{\references}{
    \reference{Guidance from the Health and Safety Executive, including manual
    handling procedures. \\
    \url{http://www.hse.gov.uk/risk/index.htm}}

    \reference{H\&S guidance from the Union Southampton website. \\
    \url{https://www.unionsouthampton.org/groups/admin/howto/protection}}

    \reference{Risk assessments prepared for events run by Student Robotics in 2015. \\
    \url{https://github.com/srobo-southampton/risk-assessments/tree/master/old}}

    \reference{Lithium polymer battery charging procedure, available on the
    Student Robotics website. \\
    \url{https://www.studentrobotics.org/docs/kit/batteries/imax_b6_charger\#ChargingChecklist}}
}


\newcommand{\risks}{
    \risk
        {Manual handling of heavy objects}
        {Competitors or mentors could experience minor injury or back pains
         resulting from improper lifting methods.}
        {\item The HSE manual handling guidelines\manualhandlingfootnote are to
         be followed for all tasks involving heavy lifting.}
        {\item No further action required.}
        {1} % Likelihood (/3)
        {2} % Impact (/3)

    \risk
        {Interaction with autonomous robots}
        {Competitors or mentors could encounter minor injuries if the robots
         move unexpectedly.}
        {\item Robots are only to be tested under supervision.
         \item When robots are switched on, they will be treated as though they
         could become active at any moment.}
        {\item The ESO\esofootnote will verify that the robots do not present
         any sharp edges. If any are found, they will be remove, covered, or
         otherwise modified to reduce the chance and severity of injury they
         could cause.}
        {1} % Likelihood (/3)
        {1} % Impact (/3)

    \risk
        {Use of manual and/or power tools}
        {Competitors or mentors could experience minor injury as a result of an
         accident or through improper use of tools.}
        {\item Care will be taken with tools to ensure that minimal injury
         results in the event of an accident.
         \item Power tools are only to be used by competitors while under
         supervision by a responsible adult.}
        {\item No further action required.}
        {2} % Likelihood (/3)
        {1} % Impact (/3)

    \risk
        {Soldering}
        {Competitors may suffer burns through inappropriate use of soldering
         irons. Soldering may produce fumes which can lead to asthma.}
        {\item All soldering irons are to be treated as if they are hot even if
         they are unplugged (since they may still be cooling down).
         \item Soldering will only be permitted in rooms with appropriate
         ventilation.
         \item Safety glasses are to be worn when soldering.}
        {\item No further action required.}
        {1} % Likelihood (/3)
        {2} % Impact (/3)

    \risk
        {Electrical equipment (robots, computers, battery chargers,
         stage lighting/sound equipment)}
        {Competitors or mentors could get electrical shocks or burns from faulty
         equipment.}
        {\item Computing equipment, battery chargers and other mains-powered
         equipment have been PAT tested.
         \item Food and water are to be kept away from electrical equipment.}
        {\item The ESO will verify that all wiring in the robots is sufficiently
         insulated before they are allowed to be tested.}
        {1} % Likelihood (/3)
        {2} % Impact (/3)

    \risk
        {Obstacles on the floor, such as bags, boxes or trailing cables}
        {Competitors or mentors may suffer injury as a result of tripping.}
        {\item Cables (such as laptop power supplies) will be routed underneath
         desks wherever possible.
         \item Cables that cannot be routed under desks will be clearly marked
         to increase their visibility.
         \item Bags, boxes and other items that are potential trip hazards will
         be stacked neatly by the walls or underneath desks whenever possible.}
        {\item No further action required.}
        {2} % Likelihood (/3)
        {1} % Impact (/3)

    \risk
        {Lithium polymer batteries}
        {LiPo batteries can ignite if damaged or misused, resulting in
         smoke/fire.}
        {\item Boxes containing batteries are clearly labelled as such and will
         be handled with care at all times.
         \item Batteries will be routinely inspected by mentors for signs of
         damage or swelling, and set aside for safe disposal if necessary.
         \item Batteries are only to be charged by trained mentors. The battery
         charging procedure\chargingfootnote is to be followed at all times.}
        {\item No further action required.}
        {1} % Likelihood (/3)
        {2} % Impact (/3)
}


\newcommand{\postrisks}{
    \subsection*{Risk of fire}

    To minimise the risk of fire resulting from this activity, food and drink
    will not be allowed near electrical equipment, and naked flames will be
    prohibited. The risk of fire occurring elsewhere in the building(s) is
    controlled primarily by the building operator\estatesfacilitiesfootnote.
    The ESO will ensure that all people present are informed of the locations of
    the exits and whether any fire drills are expected to take place.
    Should a fire break out (or any other event requiring evacuation), all
    people are to evacuate through the nearest accessible exit.
}


% This file specifies the risk assessment format. It is expected that users will
% define the following macros to specify the content of the risk assessment, and
% then \input this file to generate a document using these macros.
%   \groupname - name of student group
%   \assessorname - name of risk assessor
%   \assessoremail - email address of risk assessor
%   \assessmentdate - date risk assessment was completed
%   \activityname - title of event
%   \activitydate - date(s) of event
%   \activitytime - time(s) of event
%   \activitylocation - location(s) of event
%   \activitysummary - summary of event
%   \references - links to additional documents
%   \risks - list of hazards and their control measures
% See example.tex for a more complete description of these macros.

\documentclass[a4paper,landscape]{article}

\usepackage{array}
\usepackage{booktabs}
\usepackage{calc}
\usepackage{footnote}
\usepackage[margin=2cm]{geometry}
\usepackage{hyperref}
\usepackage{intcalc}
\usepackage{longtable}
\usepackage{multicol}

% Allow \footnote to be used inside tables.
\makesavenoteenv{tabular*}

% Define a way to render hyperlinked email addresses.
\newcommand{\email}[1]{\href{mailto:#1}{#1}}

% No paragraph indentation, 5pt spacing between paragraphs.
\setlength{\parskip}{5pt}
\setlength{\parindent}{0pt}

% Set the gutter between text columns to be 1.5cm.
\setlength{\columnsep}{1.5cm}

% Title, author and date of this document.
\newcommand{\doctitle}{Risk Assessment: \activityname}
\newcommand{\docauthor}{\assessorname}
\newcommand{\docdate}{\assessmentdate}
\title{\doctitle}
\author{\docauthor}
\date{\docdate}

% PDF output parameters.
\hypersetup{
    unicode=true,
    pdftitle={\doctitle},
    pdfauthor={\docauthor}
}

\begin{document}

% The title.
{
    \centering
    % Use size 28 font. 1.05x gives 29.4pt line spacing.
    \fontsize{28pt}{29.4pt} \selectfont
    \doctitle\\
}

\vspace{25pt}

% Set out the assessment details, activity details and references sections in a
% two-column layout.
\raggedcolumns
\begin{multicols*}{2}

\section*{Assessment Details}

\begin{tabular*}{\linewidth}[c]{p{3cm}p{\linewidth-3cm}}
    \textbf{Student group:} & \groupname \\
    \textbf{Assessor name:} & \assessorname \\
    \textbf{Assessor email:} & \email{\assessoremail} \\
    \textbf{Assessment date:} & \assessmentdate \\
\end{tabular*}

\section*{References}

% This command is used inside the user-defined \references command to
% encapsulate the formatting.
\newcommand{\reference}[1]{\item #1}

Additional documents or other sources of information that were referred to when
preparing this risk assessment include:
\begin{itemize}
    \references
\end{itemize}

\columnbreak

\section*{Activity Details}

\begin{tabular*}{\linewidth}[c]{p{2cm}p{\linewidth-2cm}}
    \textbf{Date(s):} & \activitydate \\
    \textbf{Time(s):} & \activitytime \\
    \textbf{Location(s):} & \activitylocation \\
    \textbf{Summary:} & \activitysummary \\
\end{tabular*}

\end{multicols*}

% Begin risks table on a new page.
\newpage

\section*{Risks}

% This command is used inside the user-defined \risks command to encapsulate the
% formatting. Arguments are:
%   #1: hazard
%   #2: who may be affected and how
%   #3: list of control measures in place (inserted into an itemize environment)
%   #4: list of additional control measures to put in place
%   #5: likelihood level
%   #6: impact level
\newcommand{\risk}[6]{
    #1 &
    #2 &
    \vspace{-6mm}
    \begin{itemize}
        \setlength{\itemsep}{0pt plus 1pt}
        #3
    \end{itemize} &
    \vspace{-6mm}
    \begin{itemize}
        \setlength{\itemsep}{0pt plus 1pt}
        #4
    \end{itemize} &
    \intcalcMul{#5}{#6} \\
}

\begin{longtable}{>{\raggedright}p{4cm}%
                  p{5cm}%
                  p{6cm}%
                  p{6cm}%
                  p{2.2cm}}
    \toprule
    Hazard &
    Who may be affected and how &
    Control measures in place &
    Additional control measures &
    Risk level (/9) \\
    \midrule
    \endhead
    \risks
    \bottomrule
\end{longtable}

\newpage

\section*{Review}

% A table with spaces for up to three reviewers to sign and add comments.
\begin{tabular}{|p{6cm}|p{10cm}|p{4cm}|p{3cm}|}
    \hline
    Reviewer name/role &
    Comments &
    Signed &
    Date \\
    \hline
    & & & \\[1.5cm]
    \hline
    & & & \\[1.5cm]
    \hline
    & & & \\[1.5cm]
    \hline
\end{tabular}

% Description of likelihood levels, impact levels and risk levels.
\section*{Assessment Guidance}

Each hazard is assigned a number between 1 and 3 indicating the likelihood of
the hazard affecting a person:
\begin{description}
    \item[Low (1):] May only occur in exceptional circumstances
    \item[Medium (2):] Might occur in some circumstances
    \item[High (3):] Will likely occur in most circumstances
\end{description}

Similarly, each hazard is assigned a number between 1 and 3 indicating the
magnitude of the impact that the hazard would have, if it did occur:
\begin{description}
    \item[Low (1):] Superficial or minor injury. Can usually be handled by local
    first aid procedures.
    \item[Medium (2):] Serious injury, possibly resulting in hospitalisation for
    up to three days. Complete recovery/rehabilitation could take several months.
    \item[High (3):] Major or fatal injury. Requires extensive medical treatment,
    including at least three days in hospital.
\end{description}

The hazard's \emph{risk level} is then calculated to be the likelihood rating
multiplied by the impact rating. For example, a hazard that is likely to occur
in almost all circumstances but only results in a minor injury would have a
likelihood rating of $3$, an impact rating of $1$, and an overall risk level of
$3 \times 1 = 3$.

These guidelines are based on those provided in Union Southampton's risk
assessment template.

\end{document}

