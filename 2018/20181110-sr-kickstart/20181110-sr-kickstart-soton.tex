% Footnotes used inline in the text.
\newcommand{\manualhandlingfootnote}{\footnote{\url{http://www.hse.gov.uk/pubns/indg143.pdf}}}
\newcommand{\chargingfootnote}{\footnote{\url{http://docs.sourcebots.co.uk/kit/batteries/charging/}}}
\newcommand{\estatesfacilitiesfootnote}{\footnote{Estates and Facilities: \url{http://www.southampton.ac.uk/estates/}}}
\newcommand{\hscfootnote}{\footnote{Health and Safety Coordinator; for the duration of this event, this role will be filled by Andrew Barrett-Sprot (\email{abarrett-sprot@studentrobotics.org}).}}


\newcommand{\groupname}{Student Robotics}
\newcommand{\assessorname}{Andrew Barrett-Sprot}
\newcommand{\assessoremail}{abarrett-sprot@studentrobotics.org}
\newcommand{\assessmentdate}{October 20, 2018}


\newcommand{\activityname}{Student Robotics 2019 - Kickstart}
\newcommand{\activitydate}{November 10, 2018}
\newcommand{\activitytime}{All day}
\newcommand{\activitylocation}{
    Lecture theatre 32/1015,
    Computer rooms 25/1007, 25/1011, 25/1009
}
\newcommand{\activitysummary}{
    This is the opening event to a five-month robotics competition in  which
    teams of students from schools will compete to design, build and test
    autonomous robots. The robots must perform the simple task of locating and
    moving marked boxes around an arena.

    The students are aged 16--18 and will be supervised by at least one teacher
    from their school at all times.

    The programme is run by the charity Student Robotics, who
    will be considered event staff at this and future events.

    The schedule of the event is:

    \begin{itemize}
        \item Talk: 10:00--12:00 in lecture theatre 32/1015.
        \item Lunch: 12:00--12:45 in lecture theatre 32/1015.
        \item Talk: 12:45--13:30 in lecture theatre 32/1015.
        \item Workshop session: 13:30--18:00 in B25 computer rooms.
    \end{itemize}

    The workshop session involves the teams working through a set of exercises
    to familiarise themselves with the robotics kit, with technical assistance
    from SourceBots volunteers.
}


\newcommand{\references}{
    \reference{Guidance from the Health and Safety Executive, including manual
    handling procedures. \\
    \url{http://www.hse.gov.uk/risk/index.htm}}

    \reference{H\&S guidance from the Union Southampton website. \\
    \url{https://www.unionsouthampton.org/groups/admin/howto/protection}}

    \reference{Lithium polymer battery charging procedure, available on the
    SourceBots website. \\
    \url{http://docs.sourcebots.co.uk/kit/batteries/charging/}}

    \reference{Risk assessments prepared for previous events of a similar
    nature run by us.}
}


\newcommand{\risks}{
    \risk
        {Manual handling of heavy objects}
        {Competitors or staff could experience minor injury or back pains
         resulting from improper lifting methods.}
        {\item The HSE manual handling guidelines\manualhandlingfootnote are to
         be followed for all tasks involving heavy lifting.}
        {\item No further action required.}
        {1} % Likelihood (/3)
        {2} % Impact (/3)

    \risk
        {Obstacles on the floor, such as bags, boxes or trailing cables}
        {Competitors or staff may suffer injury as a result of tripping.}
        {\item Cables (such as laptop power supplies) will be routed underneath
         desks wherever possible.
         \item Cables that cannot be routed under desks will be clearly marked
         to increase their visibility.
         \item Bags, boxes and other items that are potential trip hazards will
         be stacked neatly by the walls whenever possible.}
        {\item No further action required.}
        {2} % Likelihood (/3)
        {1} % Impact (/3)

    \risk
        {Lithium polymer batteries}
        {LiPo batteries can ignite if damaged or misused, resulting in
         smoke/fire.}
        {\item Boxes containing batteries are clearly labelled as such and will
         be handled with care at all times.
         \item Batteries will be routinely inspected by staff for signs of
         damage or swelling, and set aside for safe disposal if necessary.
         \item Batteries are only to be charged in accordance with the recorded
         charging procedure\chargingfootnote.}
        {\item No further action required.}
        {1} % Likelihood (/3)
        {2} % Impact (/3)

    \risk
        {Interaction with autonomous robots/robotics equipment}
        {Competitors or staff could encounter minor injuries if the robots'
         actuators move unexpectedly.}
        {\item When robotics equipment is switched on, it will be treated as
         though its actuators could become active at any moment.}
        {\item The HSC\hscfootnote will verify that the robotics equipment does
         not present any sharp edges. If any are found, they will be removed,
         covered, or otherwise modified to reduce the chance and severity of
         injury they could cause.}
        {1} % Likelihood (/3)
        {1} % Impact (/3)

    \risk
        {Electrical equipment (robots, computers, battery chargers)}
        {Competitors or staff could get electrical shocks or burns from faulty
         equipment.}
        {\item Computing equipment, battery chargers and other mains-powered
         equipment has been PAT tested.
         \item Opened food and drink is prohibited in the vicinity of
         computers and robotics equipment.}
        {\item The HSC will verify that all wiring done by teams or already
         preexisting in the robotics kit is sufficiently insulated and robust
         before the robotics kit is allowed to be switched on.}
        {1} % Likelihood (/3)
        {2} % Impact (/3)

    \risk
        {Use of manual tools}
        {Competitors or staff could experience minor injury as a result of an
         accident or through improper use of tools.}
        {\item Care will be taken with tools to ensure that minimal injury
         results in the event of an accident.}
        {\item No further action required.}
        {2} % Likelihood (/3)
        {1} % Impact (/3)
}


\newcommand{\postrisks}{
    \subsection*{Risk of fire}

    To minimise the risk of fire resulting from this activity, food and drink
    will not be allowed near electrical equipment, and naked flames will be
    prohibited. The risk of fire occurring elsewhere in the building(s) is
    controlled primarily by the building operator\estatesfacilitiesfootnote.
    The HSC will ensure that all people present are informed of the
    locations of the exits and whether any fire drills are expected to take
    place. Should a fire break out (or any other event requiring evacuation),
    all people are to evacuate through the nearest accessible exit.
}


% This file specifies the risk assessment format. It is expected that users will
% define the following macros to specify the content of the risk assessment, and
% then \input this file to generate a document using these macros.
%   \groupname - name of student group
%   \assessorname - name of risk assessor
%   \assessoremail - email address of risk assessor
%   \assessmentdate - date risk assessment was completed
%   \activityname - title of event
%   \activitydate - date(s) of event
%   \activitytime - time(s) of event
%   \activitylocation - location(s) of event
%   \activitysummary - summary of event
%   \references - links to additional documents
%   \risks - list of hazards and their control measures
% See example.tex for a more complete description of these macros.

\documentclass[a4paper,landscape]{article}

\usepackage{array}
\usepackage{booktabs}
\usepackage{calc}
\usepackage{footnote}
\usepackage[margin=2cm]{geometry}
\usepackage{hyperref}
\usepackage{intcalc}
\usepackage{longtable}
\usepackage{multicol}

% Allow \footnote to be used inside tables.
\makesavenoteenv{tabular*}

% Define a way to render hyperlinked email addresses.
\newcommand{\email}[1]{\href{mailto:#1}{#1}}

% No paragraph indentation, 5pt spacing between paragraphs.
\setlength{\parskip}{5pt}
\setlength{\parindent}{0pt}

% Set the gutter between text columns to be 1.5cm.
\setlength{\columnsep}{1.5cm}

% Title, author and date of this document.
\newcommand{\doctitle}{Risk Assessment: \activityname}
\newcommand{\docauthor}{\assessorname}
\newcommand{\docdate}{\assessmentdate}
\title{\doctitle}
\author{\docauthor}
\date{\docdate}

% PDF output parameters.
\hypersetup{
    unicode=true,
    pdftitle={\doctitle},
    pdfauthor={\docauthor}
}

\begin{document}

% The title.
{
    \centering
    % Use size 28 font. 1.05x gives 29.4pt line spacing.
    \fontsize{28pt}{29.4pt} \selectfont
    \doctitle\\
}

\vspace{25pt}

% Set out the assessment details, activity details and references sections in a
% two-column layout.
\raggedcolumns
\begin{multicols*}{2}

\section*{Assessment Details}

\begin{tabular*}{\linewidth}[c]{p{3cm}p{\linewidth-3cm}}
    \textbf{Student group:} & \groupname \\
    \textbf{Assessor name:} & \assessorname \\
    \textbf{Assessor email:} & \email{\assessoremail} \\
    \textbf{Assessment date:} & \assessmentdate \\
\end{tabular*}

\section*{References}

% This command is used inside the user-defined \references command to
% encapsulate the formatting.
\newcommand{\reference}[1]{\item #1}

Additional documents or other sources of information that were referred to when
preparing this risk assessment include:
\begin{itemize}
    \references
\end{itemize}

\columnbreak

\section*{Activity Details}

\begin{tabular*}{\linewidth}[c]{p{2cm}p{\linewidth-2cm}}
    \textbf{Date(s):} & \activitydate \\
    \textbf{Time(s):} & \activitytime \\
    \textbf{Location(s):} & \activitylocation \\
    \textbf{Summary:} & \activitysummary \\
\end{tabular*}

\end{multicols*}

% Begin risks table on a new page.
\newpage

\section*{Risks}

% This command is used inside the user-defined \risks command to encapsulate the
% formatting. Arguments are:
%   #1: hazard
%   #2: who may be affected and how
%   #3: list of control measures in place (inserted into an itemize environment)
%   #4: list of additional control measures to put in place
%   #5: likelihood level
%   #6: impact level
\newcommand{\risk}[6]{
    #1 &
    #2 &
    \vspace{-6mm}
    \begin{itemize}
        \setlength{\itemsep}{0pt plus 1pt}
        #3
    \end{itemize} &
    \vspace{-6mm}
    \begin{itemize}
        \setlength{\itemsep}{0pt plus 1pt}
        #4
    \end{itemize} &
    \intcalcMul{#5}{#6} \\
}

\begin{longtable}{>{\raggedright}p{4cm}%
                  p{5cm}%
                  p{6cm}%
                  p{6cm}%
                  p{2.2cm}}
    \toprule
    Hazard &
    Who may be affected and how &
    Control measures in place &
    Additional control measures &
    Risk level (/9) \\
    \midrule
    \endhead
    \risks
    \bottomrule
\end{longtable}

\newpage

\section*{Review}

% A table with spaces for up to three reviewers to sign and add comments.
\begin{tabular}{|p{6cm}|p{10cm}|p{4cm}|p{3cm}|}
    \hline
    Reviewer name/role &
    Comments &
    Signed &
    Date \\
    \hline
    & & & \\[1.5cm]
    \hline
    & & & \\[1.5cm]
    \hline
    & & & \\[1.5cm]
    \hline
\end{tabular}

% Description of likelihood levels, impact levels and risk levels.
\section*{Assessment Guidance}

Each hazard is assigned a number between 1 and 3 indicating the likelihood of
the hazard affecting a person:
\begin{description}
    \item[Low (1):] May only occur in exceptional circumstances
    \item[Medium (2):] Might occur in some circumstances
    \item[High (3):] Will likely occur in most circumstances
\end{description}

Similarly, each hazard is assigned a number between 1 and 3 indicating the
magnitude of the impact that the hazard would have, if it did occur:
\begin{description}
    \item[Low (1):] Superficial or minor injury. Can usually be handled by local
    first aid procedures.
    \item[Medium (2):] Serious injury, possibly resulting in hospitalisation for
    up to three days. Complete recovery/rehabilitation could take several months.
    \item[High (3):] Major or fatal injury. Requires extensive medical treatment,
    including at least three days in hospital.
\end{description}

The hazard's \emph{risk level} is then calculated to be the likelihood rating
multiplied by the impact rating. For example, a hazard that is likely to occur
in almost all circumstances but only results in a minor injury would have a
likelihood rating of $3$, an impact rating of $1$, and an overall risk level of
$3 \times 1 = 3$.

These guidelines are based on those provided in Union Southampton's risk
assessment template.

\end{document}

