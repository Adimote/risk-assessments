% ================================================
% ============== ASSESSMENT DETAILS ==============
% ================================================

% The name of the student society.
\newcommand{\groupname}{Student Robotics (Southampton)}

% The name and email address of the person preparing this risk assessment.
\newcommand{\assessorname}{Kier Davis}
\newcommand{\assessoremail}{me@kierdavis.com}

% The date on which this risk assessment was completed.
\newcommand{\assessmentdate}{Jan 1, 2000}


% ================================================
% =============== ACTIVITY DETAILS ===============
% ================================================

% The name of the activity (either an event, or "General Activities").
\newcommand{\activityname}{World Robot Appreciation Day}

% The date(s) and time(s) of the activity.
% For "General Activities", these may be "Wednesday afternoons" and
% "17:00--23:00", for example.
\newcommand{\activitydate}{Jan 1, 2000}
\newcommand{\activitytime}{All day}

% The location the activity takes place in.
% The seminar room in Mountbatten (where the broomcupboard is) is 53/4025.
\newcommand{\activitylocation}{Room 53/4025 (Mountbatten building)}

% A summary of the activity. It should describe:
%   - what groups of people are attending, and approximate numbers of people in
%     each group
%   - what tasks will be done
%   - what equipment will be involved
%   - any other relevant information
\newcommand{\activitysummary}{
    % ...
}


% ================================================
% ================== REFERENCES ==================
% ================================================

% A list of external sources of information that were referred to when preparing
% this document. A couple of examples are given here.
\newcommand{\references}{
    \reference{Guidance from the Health and Safety Executive. \\
    \url{http://www.hse.gov.uk/risk/index.htm}}
    
    \reference{H\&S guidance from the Union Southampton website. \\
    \url{https://www.unionsouthampton.org/groups/admin/howto/protection}}
    
    % ...
}


% ================================================
% ==================== RISKS =====================
% ================================================

% A list of hazards, their control measures and their risk levels.
% Likelihood levels:
%   1 - May only occur in exceptional circumstances
%   2 - Might occur in some circumstances
%   3 - Will likely occur in most circumstances
% Impact levels:
%   1 - Superficial or minor injury. Can usually be handled by local first aid
%       procedures.
%   2 - Serious injury, possibly resulting in hospitalisation for up to thre
%       days. Complete recovery/rehabilitation could take several months.
%   3 - Major or fatal injury. Requires extensive medical treatment, including
%       at least three days in hospital.
\newcommand{\risks}{
    \risk
        {<a hazardous task or object>}
        {<who may be affected by this hazard, and in what ways>}
        {\item <a control measure already in place>
         \item <another control measure already in place>
         \item ...}
        {\item <an additional control measure that needs to be put in place>
         \item <another additional control measure that needs to be put in place>
         \item ...}
        {<likelihood level>}
        {<impact level>}
    
    % ...
}


% Ensure that this points to template.tex, which is found in the
% repository root.
% This file specifies the risk assessment format. It is expected that users will
% define the following macros to specify the content of the risk assessment, and
% then \input this file to generate a document using these macros.
%   \groupname - name of student group
%   \assessorname - name of risk assessor
%   \assessoremail - email address of risk assessor
%   \assessmentdate - date risk assessment was completed
%   \activityname - title of event
%   \activitydate - date(s) of event
%   \activitytime - time(s) of event
%   \activitylocation - location(s) of event
%   \activitysummary - summary of event
%   \references - links to additional documents
%   \risks - list of hazards and their control measures
% See example.tex for a more complete description of these macros.

\documentclass[a4paper,landscape]{article}

\usepackage{array}
\usepackage{booktabs}
\usepackage{calc}
\usepackage{footnote}
\usepackage[margin=2cm]{geometry}
\usepackage{hyperref}
\usepackage{intcalc}
\usepackage{longtable}
\usepackage{multicol}

% Allow \footnote to be used inside tables.
\makesavenoteenv{tabular*}

% Define a way to render hyperlinked email addresses.
\newcommand{\email}[1]{\href{mailto:#1}{#1}}

% No paragraph indentation, 5pt spacing between paragraphs.
\setlength{\parskip}{5pt}
\setlength{\parindent}{0pt}

% Set the gutter between text columns to be 1.5cm.
\setlength{\columnsep}{1.5cm}

% Title, author and date of this document.
\newcommand{\doctitle}{Risk Assessment: \activityname}
\newcommand{\docauthor}{\assessorname}
\newcommand{\docdate}{\assessmentdate}
\title{\doctitle}
\author{\docauthor}
\date{\docdate}

% PDF output parameters.
\hypersetup{
    unicode=true,
    pdftitle={\doctitle},
    pdfauthor={\docauthor}
}

\begin{document}

% The title.
{
    \centering
    % Use size 28 font. 1.05x gives 29.4pt line spacing.
    \fontsize{28pt}{29.4pt} \selectfont
    \doctitle\\
}

\vspace{25pt}

% Set out the assessment details, activity details and references sections in a
% two-column layout.
\raggedcolumns
\begin{multicols*}{2}

\section*{Assessment Details}

\begin{tabular*}{\linewidth}[c]{p{3cm}p{\linewidth-3cm}}
    \textbf{Student group:} & \groupname \\
    \textbf{Assessor name:} & \assessorname \\
    \textbf{Assessor email:} & \email{\assessoremail} \\
    \textbf{Assessment date:} & \assessmentdate \\
\end{tabular*}

\section*{References}

% This command is used inside the user-defined \references command to
% encapsulate the formatting.
\newcommand{\reference}[1]{\item #1}

Additional documents or other sources of information that were referred to when
preparing this risk assessment include:
\begin{itemize}
    \references
\end{itemize}

\columnbreak

\section*{Activity Details}

\begin{tabular*}{\linewidth}[c]{p{2cm}p{\linewidth-2cm}}
    \textbf{Date(s):} & \activitydate \\
    \textbf{Time(s):} & \activitytime \\
    \textbf{Location(s):} & \activitylocation \\
    \textbf{Summary:} & \activitysummary \\
\end{tabular*}

\end{multicols*}

% Begin risks table on a new page.
\newpage

\section*{Risks}

% This command is used inside the user-defined \risks command to encapsulate the
% formatting. Arguments are:
%   #1: hazard
%   #2: who may be affected and how
%   #3: list of control measures in place (inserted into an itemize environment)
%   #4: list of additional control measures to put in place
%   #5: likelihood level
%   #6: impact level
\newcommand{\risk}[6]{
    #1 &
    #2 &
    \vspace{-6mm}
    \begin{itemize}
        \setlength{\itemsep}{0pt plus 1pt}
        #3
    \end{itemize} &
    \vspace{-6mm}
    \begin{itemize}
        \setlength{\itemsep}{0pt plus 1pt}
        #4
    \end{itemize} &
    \intcalcMul{#5}{#6} \\
}

\begin{longtable}{>{\raggedright}p{4cm}%
                  p{5cm}%
                  p{6cm}%
                  p{6cm}%
                  p{2.2cm}}
    \toprule
    Hazard &
    Who may be affected and how &
    Control measures in place &
    Additional control measures &
    Risk level (/9) \\
    \midrule
    \endhead
    \risks
    \bottomrule
\end{longtable}

\newpage

\section*{Review}

% A table with spaces for up to three reviewers to sign and add comments.
\begin{tabular}{|p{6cm}|p{10cm}|p{4cm}|p{3cm}|}
    \hline
    Reviewer name/role &
    Comments &
    Signed &
    Date \\
    \hline
    & & & \\[1.5cm]
    \hline
    & & & \\[1.5cm]
    \hline
    & & & \\[1.5cm]
    \hline
\end{tabular}

% Description of likelihood levels, impact levels and risk levels.
\section*{Assessment Guidance}

Each hazard is assigned a number between 1 and 3 indicating the likelihood of
the hazard affecting a person:
\begin{description}
    \item[Low (1):] May only occur in exceptional circumstances
    \item[Medium (2):] Might occur in some circumstances
    \item[High (3):] Will likely occur in most circumstances
\end{description}

Similarly, each hazard is assigned a number between 1 and 3 indicating the
magnitude of the impact that the hazard would have, if it did occur:
\begin{description}
    \item[Low (1):] Superficial or minor injury. Can usually be handled by local
    first aid procedures.
    \item[Medium (2):] Serious injury, possibly resulting in hospitalisation for
    up to three days. Complete recovery/rehabilitation could take several months.
    \item[High (3):] Major or fatal injury. Requires extensive medical treatment,
    including at least three days in hospital.
\end{description}

The hazard's \emph{risk level} is then calculated to be the likelihood rating
multiplied by the impact rating. For example, a hazard that is likely to occur
in almost all circumstances but only results in a minor injury would have a
likelihood rating of $3$, an impact rating of $1$, and an overall risk level of
$3 \times 1 = 3$.

These guidelines are based on those provided in Union Southampton's risk
assessment template.

\end{document}

